\documentclass{article}

\usepackage{amsmath}
\usepackage[portuguese]{babel}
\usepackage{minted}

\title{Trabalho II --- Sistemas Operacionais I}
\author{João Gabriel Trombeta\\
        João Paulo Taylor Ienczak Zanette\\
        Marco Aurélio Ludwig Moraes}
\date{\today}

\begin{document}

\maketitle

\section{Resumo das alterações}

\subsection{\texttt{pm.h} e \texttt{pm.c}}

Primeiramente, para facilitar, aproveita-se o recurso de \texttt{typedef} para
melhorar a tipagem de \texttt{struct process}, dando o apelido
\texttt{process\_t}:

\inputminted[firstline=211, lastline=211]{c}{../../include/nanvix/pm.h}

Apenas foi adicionada uma nova função \texttt{cpu\_time(process\_t*)}, que
retorna o uso de CPU com implementação bem simples:

\inputminted[firstline=129,lastline=131]{c}{../../src/kernel/pm/pm.c}

\subsection{\texttt{paging.c}}

A modificação principal é nas proximidades da função \texttt{allocf(void)}.

Inicialmente, define-se uma constante \texttt{tau} ($\tau$\cite{Carr:1981},
equivalente ao \textit{T} apresentado nos slides de WSClock). Definimos seu
valor para $4000$ ciclos de relógio, referentes a uma base de $\theta =
2ms$\cite{Carr:1981}, rodando em um processador de 2GHz (eq.\ref{kkk}).

\begin{align}
    \label{kkk}
    \tau &= \frac{\theta . freq}{1000} \\
         &= \frac{2.2.10^{6}}{1000} \nonumber \\
         &= 4.10^{3} \nonumber \\
         &= 4000 \nonumber
\end{align}

Também para facilitar, chamamos a \texttt{struct <sem nome>} de \texttt{struct
\_\_frame}, a fim de poder fazer um apelido (\texttt{typedef}) também:

\inputminted[firstline=287,lastline=289]{c}{../../src/kernel/mm/paging.c}

\subsubsection{Função \texttt{allocf(void)}}

Primeiramente, foi criado um índice \texttt{i} com tempo de vida de programa
(ou seja, \texttt{static}) dentro da função a fim de simbolizar o ponteiro do
relógio.

Em seguida, foram declaradas as \textit{flags}:

\begin{description}
    \item[\texttt{cycle}]: para verificar se ocorreu um ciclo ao percorrer a
        lista de molduras --- inicialmente partindo de que não ocorreram
        ciclos;
    \item[\texttt{written}]: para verificar se ocorreram escritas durante um
        ciclo;
    \item[\texttt{owned}]: para verificar se em algum momento foi encontrada
        uma moldura pertencente ao processo atual.
\end{description}

Para o controle de ocorrência de ciclos, utiliza-se uma variável
\texttt{begin}.

Para evitar acessar um índice inválido da lista de \textit{frames} (por
incremento do ponteiro do relógio), é comparado se \texttt{i} é maior ou igual
\texttt{NR\_FRAMES}, e em caso positivo, reinicia-se o ponteiro para o primeiro
\textit{frame} (\texttt{i = 0}). Dessa forma, a lista de frames funciona de
forma virtualmente circular.

Tenta-se então procurar o \textit{frame} ideal para alocar, seguindo o
algoritmo do WSClock. Os passos são:

\begin{enumerate}
    \item Se o \textit{frame} no ponteiro não for referenciado por página
        alguma, ele é escolhido (alocação global).
    \item Caso contrário, verifica-se se o \textit{frame} no ponteiro pertence
        ao processo atual (que requisita a alocação). Em caso positivo, se não
        for um \textit{frame} compartilhado, então:
        \begin{enumerate}
            \item Marca-se \texttt{owned} como verdadeiro, simbolizando que há
                alguma página pertencente ao processo atual;
            \item Verifica-se se essa moldura foi referenciada desde a última
                interrupção. Caso tenha sido, a \textit{flag} de acessada é
                alterada para falso e atualiza-se o momento de último uso para
                o tempo atual do processo (\texttt{cpu\_time(curr\_proc)}).
            \item Caso não tenha sido referenciada, calcula-se a idade seguindo
                a eq.\ref{age}.
            \item Se a idade for maior do que a constante $\tau$, verifica-se a
                necessidade de escrevê-la em disco. Em caso positivo para
                ambos, escreve-se no disco e, se a escrita for bem sucedida,
                tiver passado um ciclo (i.e.\ uma vez por todos os
                \textit{frames}) e não houve escrita no ciclo anterior, então
                escolhe-se arbitrariamente tal moldura para alocação. Ao final
                da escrita, marca-se \texttt{written} como verdadeiro.
            \item Se a idade for maior e não há necessidade de escrita, a
                moldura é escolhida para alocação.
        \end{enumerate}
    \item Caso nenhuma moldura seja escolhida, o laço \texttt{while (TRUE)}
        ainda estará em execução. Assim, incrementa-se o ponteiro \texttt{i},
        verifica-se a necessidade de voltar ao início da lista e, caso tenha
        completado um ciclo (ou seja, se \texttt{i} voltou a ter o mesmo valor
        de \texttt{begin}), retorna-se \texttt{-1} caso não hajam
        \textit{frames} disponíveis (ou seja, nenhuma pertence ao processo
        atual e não há molduras livres), ou, caso haja alguma, marca-se que
        pelo menos um ciclo foi completado (\texttt{cycle} torna-se
        verdadeiro).
    \item Por fim, uma vez fora do ciclo, significa que uma moldura foi
        escolhida. Atualiza-se a idade da moldura para o tempo de CPU do
        processo atual e indica-se que há uma página referenciando a moldura.
        Por fim, a moldura é retornada e o ponteiro \texttt{i} é incrementado
        (e garantido de que não irá passar do limite de \textit{frames} na
        próxima execução da função, pela primeira comparação feita nela).
\end{enumerate}

\begin{align}
    \label{age}
    age_{ellapsed} &= cpu_{time}(process) - age_{frame}
\end{align}

\bibliographystyle{unsrt}
\bibliography{references}

\end{document}
