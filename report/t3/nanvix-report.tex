\documentclass{article}

\usepackage{amsmath}
\usepackage[portuguese]{babel}
\usepackage{color}
\usepackage{minted}

\definecolor{gray}{gray}{0.6}

\title{Trabalho III --- Sistemas Operacionais I}
\author{João Gabriel Trombeta\\
        João Paulo Taylor Ienczak Zanette}
\date{\today}

\begin{document}

\maketitle

\section{Resumo das alterações}

\subsection{\texttt{include/nanvix/dev.h} e \texttt{src/kernel/dev/ata/ata.c}}

Como o uso de sincronia/assincronia para o acesso ao dispositivo ATA é através
de uma flag (\texttt{REQ\_SYNC}) passada como parâmetro para a função
\texttt{ata\_sched} (indiretamente pela \texttt{ata\_sched\_buffered}, que por
suas flags vez vem de \texttt{ata\_readblk}, que está registrada pelo ponteiro
de função da estrutura de dados de dispositivo). Sendo assim, foi modificada a
assinatura da função \texttt{ata\_readblk}, adicionando um parâmetro
\texttt{sync} (\texttt{unsigned}, já que em C não há, pela linguagem, um tipo
booleano), simbolizando se a escrita será ($1$) ou não ($0$) síncrona:

\inputminted[firstline=647, lastline=647]{c}{../../src/kernel/dev/ata/ata.c}

A alteração também se reflete no ponteiro de função da estrutura \texttt{bdev}:

\inputminted[firstline=132, lastline=132]{c}{../../include/nanvix/dev.h}

Por consequência, as funções apontadas pelas instâncias de \texttt{struct bdev}
já existentes também tiveram que ser atualizadas para corroborar com a
assinatura nova. Para evitr mudanças em quem já usa a função
\texttt{bdev\_readblk}, seu código foi passado para uma nova função,
\texttt{bdev\_readblk\_syncable} (que possui um parâmetro a mais,
\texttt{sync}, que apenas é redirecionado à função associada à instância de
\texttt{struct bdev}), passando a apenas chamar essa nova função com o
parâmetro \texttt{sync} como verdadeiro ($1$), simulando o comportamento
anterior às mudanças.

\subsection{\texttt{src/kernel/dev/ramdisk/ramdisk.c}}

Para evitar que o compilador acuse que o parâmetro novo não foi utilizados nas
funções já existentes de leitura do bloco, apenas foi adicionado um
\texttt{(void) (parâmetro)}:

\inputminted[firstline=133, lastline=135]{c}{../../src/kernel/dev/ramdisk/ramdisk.c}

\subsection{\texttt{src/nanvix/fs/file.c} e \texttt{src/kernel/fs/buffer.c}}

O \textit{prefetching} foi feito apenas forçando a leitura de um bloco, mas sem
efetivamente fazer uso dele. Primeiramente, em vez de a função de leitura
requisitar a leitura de um único bloco, ela carrega $N_b$ blocos (conforme a
equação~\ref{nb-eq}), sendo $N_b - 1$ via \textit{prefetching}, através da
função \texttt{n\_bread}.

{\Large
\begin{align}
    N_b = N_{blocks\_to\_read} = 1 + \min(2, \frac{n_{bytes\_to\_read}}{BlockSize})
    \label{nb-eq}
\end{align}
}

Assim, limitam-se para no máximo 2 blocos sofrerem \textit{prefetch}. Isso é
feito no trecho (em \texttt{src/kernel/fs/file.c}):

\inputminted[firstline=305, lastline=305]{c}{../../src/kernel/fs/file.c}

Sendo que \texttt{MIN} é uma macro definida em \texttt{include/nanvix/utils.h}
que apenas retorna o menor entre dois valores:

\inputminted{c}{../../include/nanvix/utils.h}

Foi criada também uma função \texttt{breasy} (\textbf{B}lock \textbf{RE}ad
\textbf{ASY}nc), que apenas força a leitura do bloco (o que é feito chamando
\texttt{bdev\_readblk\_syncable} com \texttt{sync = FALSE} (0)), mas nada faz
com ele. Ela é utilizada em \texttt{n\_bread}, que lê \texttt{$num\_blocks -
1$} blocos de maneira assíncrona e um de maneira síncrona:

\inputminted[firstline=574, lastline=577]{c}{../../src/kernel/fs/buffer.c}

Por razões de compatibilidade de código, a antiga \texttt{bread}, então, passa
a apenas chamar \texttt{n\_bread} com número de blocos = 1.

\vspace*{\fill}
\textcolor{gray}{%
    \textit{\tiny (Em memória de Marco Aurélio Ludwig Moraes, que saiu do
    grupo.)}}

\end{document}
