\documentclass{article}

\usepackage{amsmath}
\usepackage[portuguese]{babel}

\title{Trabalho I --- Sistemas Operacionais I}
\author{João Gabriel Trombeta\\
        João Paulo Taylor Ienczak Zanette\\
        Marco Aurélio Ludwig Moraes}
\date{\today}

\begin{document}

\maketitle

\section{Resumo das alterações}

\subsection{pm.h}

De acordo com Tanenbaum, o \textit{quantum} ideal para implementação de filas
múltiplas é de 20 e 50ns. Como foram utilizadas 40 filas (uma para cada valor
de \texttt{nice}), o \textit{quantum} mínimo (definido por
\texttt{PROC\_QUANTUM}) foi ajustado para 15 e o máximo para 54, para se manter
próximo do intervalo proposto, conforme descrito no cálculo de tempo de uso de
CPU em \texttt{yield(void)}.

\subsection{pm.c}

Foi alterado o \texttt{counter} para 0, sendo, além do \textit{quantum},
responsável pelo envelhecimento do processo. Foi considerado então que quanto
maior o \texttt{counter}, mais envelhecido é o processo.

\subsection{sched.c}

Cada valor de \texttt{nice} é tratado como uma fila. Seu uso é explicado nas
funções \texttt{yield(void)} e \texttt{sched(struct process*}).

\subsubsection{Função \texttt{yield(void)}}

Percorre todos os processos para decidir qual será o próximo. Processos
advindos de uma interrupção terão sempre maior prioridade (conforme definido em
\texttt{pm.h}). Caso os processos comparados tenham a mesma prioridade, o
processo com menor \texttt{nice} (ou seja, na fila de menor índice) é atribuído
a \texttt{next} (variável que controla qual o próximo processo escolhido para
ocupar a CPU).

Todos os processos não escalonados são envelhecidos (\texttt{++counter}). Uma
vez que o \texttt{counter} atinge o valor 10, se o processo não estiver na fila
de menor índice, ele desce uma fila (aumentando sua prioridade) e seu contador
é reiniciado. Quando escalonado, o valor de \texttt{counter} precisará
representar o \textit{quantum} do processo, conforme o cálculo descrito
em~\ref{quantum-calc}. Esse cálculo respeita o que foi dito anteriormente sobre
o intervalo do \textit{quantum}, já que o menor índice de fila é 0 e há 40
filas (mantendo o intervalo $[15 + 0, 15 + 39]$).

\begin{equation}
    counter \leftarrow Quantum_{\min} + QueueIndex\\
    \label{quantum-calc}
\end{equation}

\subsubsection{Função \texttt{sched(struct process*)}}

Supondo um processo que tenha vindo do estado \textit{running} (i.e.\ seu
\textit{quantum} terminou), caso ele não esteja na fila de maior índice, ele
sobe uma fila (diminuindo sua prioridade), caso contrário ele permanece na
mesma fila.

Por outro lado, supondo um processo que tenha vindo de um \textit{sleep} ou
\textit{wait}, então sua prioridade é incrementada (descendo uma fila).

\end{document}
